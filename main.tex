% Pengaturan ukuran teks dan bentuk halaman dua sisi
\documentclass[12pt,oneside]{book}

% Pengaturan ukuran halaman dan margin
\usepackage[a4paper,top=30mm,left=30mm,right=20mm,bottom=25mm]{geometry}

% Pengaturan ukuran spasi
\usepackage[singlespacing]{setspace}

% Pengaturan caption untuk tabel
\usepackage{caption}

% Pengaturan detail pada file PDF
\usepackage[pdfauthor={\@author},bookmarksnumbered,pdfborder={0 0 0}]{hyperref}

% Pengaturan ukuran indentasi
\setlength{\parindent}{2em}

% Package lainnya
\usepackage{etoolbox} % Mengubah fungsi default

% Pengaturan jenis karakter
\usepackage[utf8]{inputenc}
\usepackage{url}

\usepackage[style=apa, backend=biber]{biblatex}
\usepackage{xurl}
\usepackage{enumitem} % Pembuatan list
\usepackage{graphicx} % Input gambar
\usepackage{longtable} % Pembuatan tabel
\usepackage[table,xcdraw]{xcolor} % Pewarnaan tabel
\usepackage{eso-pic} % Untuk menggunakan background image di halaman
\usepackage{txfonts} % Font times
\usepackage{changepage} % Pembuatan teks kolom
\usepackage{multicol} % Pembuatan kolom ganda
\usepackage{multirow} % Pembuatan baris ganda
\usepackage{tabularx} % Untuk mengatur kolom, seperti grid pada CSS
\usepackage{wrapfig}
\usepackage{float}

% Pengaturan format daftar isi, daftar gambar, dan daftar tabel
\usepackage[titles]{tocloft}
\setlength{\cftsecindent}{2em}
\setlength{\cftsubsecindent}{2em}
\setlength{\cftsubsubsecindent}{2em}
\setlength{\cftbeforechapskip}{1.5ex}
\setlength{\cftbeforesecskip}{1ex}
\setlength{\cftbeforesubsecskip}{1ex}
\setlength{\cftbeforesubsubsecskip}{1ex}
\setlength{\cftbeforetoctitleskip}{0cm}
\setlength{\cftbeforeloftitleskip}{4ex}
\setlength{\cftafterloftitleskip}{0cm}
\setlength{\cftbeforelottitleskip}{0cm}
\setlength{\cftfigindent}{0pt}
\setlength{\cfttabindent}{0pt}
\renewcommand{\cfttoctitlefont}{\hfill\Large\bfseries} % command untuk membuat heading bold dan besar
\renewcommand{\cftaftertoctitle}{\hfill}
\renewcommand{\cftloftitlefont}{\hfill\Large\bfseries}
\renewcommand{\cftafterloftitle}{\hfill}
\renewcommand{\cftlottitlefont}{\hfill\Large\bfseries}
\renewcommand{\cftafterlottitle}{\hfill}

% Definisi untuk "Hati ini sengaja dikosongkan"
\patchcmd{\cleardoublepage}{\hbox{}}{
  \thispagestyle{empty}
  \vspace*{\fill}
  \begin{center}\textit{[Halaman ini sengaja dikosongkan]}\end{center}
  \vfill}{}{}

% Pengaturan penomoran halaman
\usepackage{fancyhdr}
\fancyhf{}
\renewcommand{\headrulewidth}{0pt}
\pagestyle{fancy}
\fancyfoot[C,CO]{\thepage}
\patchcmd{\chapter}{plain}{fancy}{}{}
\patchcmd{\chapter}{empty}{plain}{}{}

% Pengaturan format judul bab
\usepackage{titlesec}
\renewcommand{\thesection}{\thechapter.\arabic{section}}
\titleformat{\chapter}[hang]{\centering\bfseries\large}{BAB\ \arabic{chapter}\ }{0ex}{\vspace{0ex}\centering}
\titleformat*{\section}{\large\bfseries}
\titleformat*{\subsection}{\normalsize\bfseries}
\titleformat*{\subsubsection}{\normalsize\bfseries}
\titlespacing{\chapter}{0ex}{0ex}{4ex}
\titlespacing{\section}{0ex}{1ex}{0ex}
\titlespacing{\subsection}{0ex}{0.5ex}{0ex}
\titlespacing{\subsubsection}{0ex}{0.5ex}{0ex}
\setcounter{secnumdepth}{4} % Untuk memberi penomoran pada \subsubsection
\setcounter{tocdepth}{4}

\counterwithin{figure}{chapter}
\counterwithin{table}{chapter}

% Mengganti figure dan table menjadi gambar dan tabel
\renewcommand{\figurename}{Gambar}
\renewcommand{\tablename}{Tabel}

% Tambahkan format tanda hubung yang benar di sini
\hyphenation{
  con-cepts
  ku-ber-ne-tes
}


% Menambahkan resource daftar pustaka
\addbibresource{pustaka/pustaka.bib}

% table related
\newcommand{\w}{}
\newcommand{\G}{\cellcolor{gray}}

% Isi keseluruhan dokumen
\begin{document}
  % Nomor halaman pembuka dimulai dari sini
  \pagenumbering{roman}

  % Atur ulang penomoran halaman
  \setcounter{page}{1}

  % Sampul Bahasa Indonesia
  \newcommand\covercontents{sampul/konten-id.tex}
  \input{sampul/sampul-luar.tex}

  % Lembar pengesahan
  \chapter*{LEMBAR PENGESAHAN}
\addcontentsline{toc}{chapter}{LEMBAR PENGESAHAN}
\thispagestyle{empty}

\begin{center}
  % Ubah kalimat berikut dengan judul tugas akhir
  \textbf{IMPLEMENTASI \emph{MULTI-TENANCY} UNTUK \emph{PROVISIONING} KLASTER KUBERNETES}
\end{center}

\begingroup
% Pemilihan font ukuran small
\small

\begin{center}
  % Ubah kalimat berikut dengan pernyataan untuk lembar pengesahan
  \textbf{PROPOSAL TUGAS AKHIR} \\
  Diajukan untuk memenuhi salah satu syarat memperoleh gelar
  Sarjana Komputer pada
  Program Studi S-1 Teknik Informatika \\
  Departemen Teknik Informatika \\
  Fakultas Teknologi Elektro dan Informatika Cerdas \\
  Institut Teknologi Sepuluh Nopember
\end{center}

\begin{center}
  % Ubah kalimat berikut dengan nama dan NRP mahasiswa
  Oleh: \textbf{Urdhanaka Aptanagi} \\
  NRP. 5025211123
\end{center}

\begin{center}
  Disetujui Oleh:
\end{center}

\vspace{10ex}

\begingroup
% Menghilangkan padding
\setlength{\tabcolsep}{0pt}

\noindent
\begin{tabularx}{\textwidth}{X c}
  % Ubah kalimat-kalimat berikut dengan nama dan NIP dosen pembimbing pertama
        &                 \\
  Royyana Muslim Ijtihadie, S.Kom., M.Kom., Ph.D.     &                 \\
  NIP: 19770824 200304 1 001                          & (Pembimbing)    \\
                                                      &                 \\
                                                      &                 \\
                                                      &                 \\
  % Ubah kalimat-kalimat berikut dengan nama dan NIP dosen pembimbing kedua
  Ary Mazharuddin Shiddiqi, S.Kom., M.Comp.Sc., Ph.D. &                 \\
  NIP: 19810620 200501 1 003                          & (Ko-Pembimbing) \\
\end{tabularx}
\endgroup

\vspace{\fill}

\begin{center}
  \textbf{SURABAYA} \\
  \textbf{Desember, 2024}
\end{center}
\endgroup


  % Abstrak
  \chapter*{ABSTRAK}
\begin{center}
  \large
  \textbf{IMPLEMENTASI \emph{MULTI-TENANCY} UNTUK \emph{PROVISIONING} KLUSTER KUBERNETES}
\end{center}
\addcontentsline{toc}{chapter}{ABSTRAK}
% Menyembunyikan nomor halaman
\thispagestyle{empty}

\begin{flushleft}
  \setlength{\tabcolsep}{0pt}
  \bfseries
  \begin{tabular}{ll@{\hspace{6pt}}l}
  Nama Mahasiswa / NRP&:& Urdhanaka Aptanagi / 5025211123\\
  Departemen&:& Teknik Informatika FTEIC - ITS\\
  Dosen Pembimbing&:& 1. Nikola Tesla, S.T., M.T.\\
  & & 2. Wernher von Braun, S.T., M.T.\\
  \end{tabular}
  \vspace{4ex}
\end{flushleft}
\textbf{Abstrak}

% Isi Abstrak
Sistem terdistribusi merupakan salah satu cara untuk meningkatkan kinerja
dan keandalan sebuah sistem. Sebuah \emph{tools} yang sering digunakan dalam sistem
terdistribusi adalah Kubernetes.

\vspace{2ex}
\noindent
\textbf{Kata Kunci: \emph{multi-tenancy, virtual cluster, kubernetes}}


  \chapter*{ABSTRACT}
\begin{center}
  \large
  \textbf{MULTI-TENANCY IMPLEMENTATION FOR KUBERNETES CLUSTER PROVISIONING} \end{center}
% Menyembunyikan nomor halaman
\thispagestyle{empty}

\begin{flushleft}
  \setlength{\tabcolsep}{0pt}
  \bfseries
  \begin{tabular}{lc@{\hspace{6pt}}l}
  Student Name / NRP&: &Urdhanaka Aptanagi / 5025211123\\
  Department&: &Informatics Engineering FTEIC - ITS\\
  Advisor&: &1. Nikola Tesla, S.T., M.T.\\
  & & 2. Wernher von Braun, S.T., M.T.\\
  \end{tabular}
  \vspace{4ex}
\end{flushleft}
\textbf{Abstract}

% Isi Abstrak
test.

\vspace{2ex}
\noindent
\textbf{Keywords: \emph{multi-tenancy, virtual cluster, kubernetes}}


  \begin{spacing}{1}
    % Daftar isi
    \renewcommand*\contentsname{DAFTAR ISI}
    \addcontentsline{toc}{chapter}{\contentsname}
    \tableofcontents

    % Daftar gambar
    \renewcommand*\listfigurename{DAFTAR GAMBAR}
    \addcontentsline{toc}{chapter}{\listfigurename}
    \listoffigures

    % Daftar tabel
    \renewcommand*\listtablename{DAFTAR TABEL}
    \addcontentsline{toc}{chapter}{\listtablename}
    \listoftables
  \end{spacing}

  % Nomor halaman isi dimulai dari sini
  \pagenumbering{arabic}

  % Konten pendahuluan
  \chapter{PENDAHULUAN}

\section{Latar Belakang}

Internet merupakan salah satu inovasi yang mengubah dunia. Internet
memungkinkan penggunanya untuk berkomunikasi, berbagi informasi, dan
mengakses berbagai layanan. Salah satu bentuk layanan tersebut adalah
layanan komputasi dimana pengguna dapat menggunakan sumber daya komputasi
yang telah disediakan sehingga pengguna tidak perlu melakukan komputasi
di perangkat mereka sendiri. Hal tersebut memungkinkan pengguna untuk
melakukan komputasi yang berat tanpa harus memiliki perangkat yang
memadai.

Pengelolaan sumber daya komputasi memerlukan manajemen yang
baik agar sumber daya tersebut dapat digunakan secara efisien dan sesuai
dengan kebutuhan pengguna. Untuk mengelola sumber daya komputasi tersebut,
salah satu \emph{tools} yang dapat digunakan adalah Kubernetes. Kubernetes
adalah sebuah sistem manajemen \emph{container} sumber terbuka.

Namun, penggunaan Kubernetes secara \emph{default} bukan berupa \emph{multi-tenant}.
Semua \emph{pods} yang merupakan unit terkecil di dalam Kubernetes dapat berinteraksi
satu sama lain dan semua \emph{network traffic} tidak terenkripsi \parencite{kubernetes-website-multi-tenancy}.
Hal tersebut dapat menimbulkan masalah karena pengguna yang satu dapat
melihat data dari pengguna yang lain. Oleh karena itu, diperlukan
implementasi \emph{multi-tenancy} pada Kubernetes jika sebuah kluster Kubernetes
akan digunakan oleh lebih dari satu pengguna.

Pada penelitian ini akan dibahas mengenai implementasi \emph{multi-tenancy}
pada Kubernetes. Implementasi tersebut diharapkan membuat kluster
Kubernetes dapat digunakan oleh beberapa pengguna sekaligus.

\section{Rumusan Masalah}

Berdasarkan hal yang telah dipaparkan di latar belakang, rumusan masalah
yang diangkat dalam Tugas Akhir ini adalah sebagai berikut:
\begin{enumerate}
  \vspace{-0.3cm}\item{Bagaimana cara mengimplementasikan \emph{multi-tenancy} pada kluster Kubernetes?}
  \vspace{-0.3cm}\item{Apa perbedaan implementasi \emph{multi-tenancy} yang dapat digunakan pada kluster Kubernetes?}
\end{enumerate}

\section{Batasan Masalah atau Ruang Lingkup}

Batasan masalah atau ruang lingkup dari penelitian ini adalah sebagai berikut:
\begin{enumerate}
  \vspace{-0.3cm}\item{asdasdad}
\end{enumerate}

\section{Tujuan}

Tujuan dari penelitian ini adalah mengimplementasikan \emph{multi-tenancy}
pada kluster Kubernetes yang dapat digunakan.

\section{Manfaat}

Manfaat dari penelitian ini adalah 


  % Konten tinjauan pustaka
  \chapter{TINJAUAN PUSTAKA}

\section{Hasil penelitian/perancangan terdahulu}

Kubernetes memiliki beberapa fitur yang memungkinkan untuk penerapan
\emph{multi-tenancy}. Salah satu fitur tersebut adalah \emph{namespace}.
\emph{Namespace} adalah fitur yang memisahkan sumber daya di dalam kluster
ke dalam beberapa bagian.

\section{Teori/Konsep Dasar}

\subsection{Kluster}

Kluster adalah konsep dasar dari Kubernetes. Kluster adalah kumpulan dari
satu atau lebih \emph{node} yang digunakan untuk menjalankan \emph{pods} yang
menjalankan aplikasi yang dikemas (\emph{containerized}). 

\subsection{\emph{Multi-tenant}}

\emph{Multi-tenant} adalah sebuah konsep dimana sebuah sistem dapat digunakan
oleh lebih dari satu pengguna atau \emph{tenant}. Dalam Kubernetes, implementasi
\emph{multi-tenant} berada di kluster sehingga sebuah kluster dapat digunakan
oleh satu atau lebih pengguna. Maka \parencite{oliva_multi-tenancy_2024}


  % Konten metodologi
  \chapter{METODOLOGI}

% Ubah konten-konten berikut sesuai dengan isi dari metodologi

\section{Metode yang digunakan}

\lipsum[11]

% Contoh input gambar dengan format *.jpg
\begin{figure} [H] \centering
  % Nama dari file gambar yang diinputkan
  \includegraphics[scale=0.45]{gambar/blueprint.jpg}
  % Keterangan gambar yang diinputkan
  \caption{\emph{Blueprint} roket yang akan diuji coba}
  % Label referensi dari gambar yang diinputkan
  \label{fig:Blueprint}
\end{figure}

% Contoh penggunaan referensi dari gambar yang diinputkan
Pada \emph{blueprint} yang tertera di Gambar \ref{fig:Blueprint}. Pada \parencite{6830928}

\section{Bahan dan peralatan yang digunakan}

\lipsum[13]


  % Daftar pustaka
  \chapter*{DAFTAR PUSTAKA}
  \addcontentsline{toc}{chapter}{DAFTAR PUSTAKA}
  \renewcommand\refname{}
  \vspace{2ex}
  \renewcommand{\bibname}{}
  \begingroup
    \def\chapter*#1{}
    \printbibliography
  \endgroup

\end{document}
