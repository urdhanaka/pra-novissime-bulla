\chapter{PENDAHULUAN}

\section{Latar Belakang}

Internet merupakan salah satu inovasi yang mengubah dunia. Internet
memungkinkan penggunanya untuk berkomunikasi, berbagi informasi, dan
mengakses berbagai layanan. Teknologi internet yang semakin maju
dan pengguna internet yang semakin banyak memunculkan banyak penyedia
jasa layanan melalui internet, salah satunya yaitu penyedia jasa komputasi.
Dengan menggunakan jasa komputasi, pengguna tidak perlu melakukan komputasi
di perangkat mereka sendiri. Dengan begitu, komputasi berat dapat dilakukan
dengan menyewa \emph{hardware} yang disediakan oleh penyedia jasa.

Semakin meningkatnya pengguna dari sebuah jasa, penyedia jasa perlu
memikirkan tentang skalabilitas dari jasa yang mereka sediakan. Hal tersebut
diperlukan agar \emph{resource} dapat digunakan secara efisien, adil, dan mampu
untuk menangani banyaknya pengguna. Salah satu \emph{tools} yang dapat digunakan
untuk skalabilitas adalah Kubernetes. Kubernetes adalah sebuah sistem manajemen
\emph{container} sumber terbuka. Kubernetes didesain untuk mengelola siklus hidup
dari \emph{container} sehingga kelangsungan sistem dapat diprediksi, \emph{scalable},
serta memiliki ketersediaan yang tinggi. Beberapa layanan yang disediakan oleh
Kubernetes adalah \emph{load balancing} dan \emph{service discovery},
orkestrasi untuk tempat penyimpanan, automasi sistem untuk \emph{rollout}
dan \emph{rollback} sistem, serta mekanisme pemulihan pada sistem.

Namun, penggunaan Kubernetes secara \emph{default} bukan berupa \emph{multi-tenant}.
Semua \emph{pods} yang merupakan unit terkecil di dalam klaster Kubernetes dapat
berinteraksi satu sama lain dan semua \emph{network traffic} tidak terenkripsi \parencite{kubernetes-website-multi-tenancy}.
Hal tersebut dapat menimbulkan masalah karena pengguna yang satu dapat
melihat data dari pengguna yang lain. Oleh karena itu, diperlukan
implementasi \emph{multi-tenancy} yang baik pada Kubernetes jika
sebuah klaster Kubernetes akan digunakan oleh lebih dari satu pengguna.

Pada penelitian ini akan dibahas mengenai implementasi \emph{multi-tenancy}
pada Kubernetes. Implementasi tersebut diharapkan membuat klaster
Kubernetes dapat digunakan oleh beberapa pengguna sekaligus dan juga dapat
mengurangi \emph{cost} dengan cara memanfaatkan \emph{resources} yang ada
pada klaster secara menyeluruh dan efektif sehingga tidak perlu menggunakan
\emph{resources} yang baru.

\section{Rumusan Masalah}

Berdasarkan hal yang telah dipaparkan di latar belakang, rumusan masalah
yang diangkat dalam Tugas Akhir ini adalah sebagai berikut:
\begin{enumerate}[itemsep=-0.2cm, topsep=-0.3cm]
  \item{Bagaimana cara mengimplementasikan \emph{multi-tenancy} pada klaster Kubernetes?}
  \item{Apa perbedaan implementasi \emph{multi-tenancy} yang dapat digunakan pada klaster Kubernetes?}
\end{enumerate}

\section{Batasan Masalah atau Ruang Lingkup}

Batasan masalah atau ruang lingkup dari penelitian ini adalah sebagai berikut:
\begin{enumerate}[itemsep=-0.2cm, topsep=-0.3cm]
  \item{Klaster Kubernetes yang digunakan berada di lokal menggunakan \emph{tools} untuk membuat klaster lokal}
  \item{Versi Kubernetes yang digunakan adalah versi v1.31.4}
  \item{Penelitian ini akan menggunakan beberapa implementasi \emph{multi-tenancy} pada Kubernetes yang pernah diciptakan sebelumnya}
\end{enumerate}

\section{Tujuan}

Tujuan dari penelitian ini adalah sebagai berikut:
\begin{enumerate}[itemsep=-0.2cm, topsep=-0.3cm]
  \item{Mengimplementasikan \emph{multi-tenancy} pada klaster Kubernetes yang dapat digunakan.}
  \item{Mengetahui perbedaan implementasi \emph{multi-tenancy} pada Kubernetes.}
\end{enumerate}

\section{Manfaat}

Manfaat dari penelitian ini adalah untuk meningkatkan kegunaan dari
klaster Kubernetes dengan menggunakan teknologi \emph{multi-tenant}
sehingga dapat mengurangi \emph{cost} dalam penyediaan
\emph{resources computing}.
