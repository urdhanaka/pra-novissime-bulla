\chapter{PENDAHULUAN}

\section{Latar Belakang}

Internet merupakan salah satu inovasi yang mengubah dunia. Internet
memungkinkan pengguna untuk berkomunikasi, berbagi informasi, dan
mengakses berbagai layanan. Salah satu bentuk layanan tersebut adalah
layanan komputasi dimana pengguna dapat menggunakan sumber daya komputasi
yang telah disediakan, sehingga pengguna tidak perlu melakukan komputasi
di perangkat mereka sendiri. Salah satu konsep yang biasa digunakan
adalah konsep \emph{containerization} dimana komputer membuat sebuah \emph{container}
yang berisi aplikasi dan semua dependensinya. \emph{}

\section{Rumusan Masalah}

% Ubah paragraf berikut sesuai dengan rumusan masalah dari tugas akhir
Berdasarkan hal yang telah dipaparkan di latar belakang, rumusan masalah
yang diangkat dalam Tugas Akhir ini adalah sebagai berikut:
\begin{enumerate}
  \vspace{-0.3cm}\item{Bagaimana membuat virtual cluster yang bisa
  menghidupkan mesin dan mematikan mesin kapan saja, bisa men-generate
  mesin secara instant (case virtual cluster pada saat ini tidak bisa seperti itu,
  mesin harus hidup setiap saat?)}
  \vspace{-0.3cm}\item{Bagaimana ...}
  \vspace{-0.3cm}\item{Bagaimana ...}
\end{enumerate}

\section{Batasan Masalah atau Ruang Lingkup}

Batasan masalah atau ruang lingkup dari penelitian ini adalah sebagai berikut:
\begin{enumerate}
  \vspace{-0.3cm}\item{Bagaimana ...}
  \vspace{-0.3cm}\item{Bagaimana ...}
\end{enumerate}

\section{Tujuan}

Tujuan dari penelitian ini adalah menerapkan ...

\section{Manfaat}

% Ubah paragraf berikut sesuai dengan tujuan penelitian dari tugas akhir
Manfaat dari penelitian ini adalah ...
