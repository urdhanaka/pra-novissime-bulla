\chapter{PENDAHULUAN}

\section{Latar Belakang}

Internet merupakan salah satu inovasi yang mengubah dunia. Internet
memungkinkan penggunanya untuk berkomunikasi, berbagi informasi, dan
mengakses berbagai layanan. Salah satu bentuk layanan tersebut adalah
layanan komputasi dimana pengguna dapat menggunakan sumber daya komputasi
yang telah disediakan sehingga pengguna tidak perlu melakukan komputasi
di perangkat mereka sendiri. Hal tersebut memungkinkan pengguna untuk
melakukan komputasi yang berat tanpa harus memiliki perangkat yang
memadai.

Pengelolaan sumber daya komputasi memerlukan manajemen yang
baik agar sumber daya tersebut dapat digunakan secara efisien dan sesuai
dengan kebutuhan pengguna. Untuk mengelola sumber daya komputasi tersebut,
salah satu \emph{tools} yang dapat digunakan adalah Kubernetes. Kubernetes
adalah sebuah sistem manajemen \emph{container} sumber terbuka.

Namun, penggunaan Kubernetes secara \emph{default} bukan berupa \emph{multi-tenant}.
Semua \emph{pods} yang merupakan unit terkecil di dalam Kubernetes dapat berinteraksi
satu sama lain dan semua \emph{network traffic} tidak terenkripsi \parencite{kubernetes-website-multi-tenancy}.
Hal tersebut dapat menimbulkan masalah karena pengguna yang satu dapat
melihat data dari pengguna yang lain. Oleh karena itu, diperlukan
implementasi \emph{multi-tenancy} pada Kubernetes jika sebuah kluster Kubernetes
akan digunakan oleh lebih dari satu pengguna.

Pada penelitian ini akan dibahas mengenai implementasi \emph{multi-tenancy}
pada Kubernetes. Implementasi tersebut diharapkan membuat kluster
Kubernetes dapat digunakan oleh beberapa pengguna sekaligus.

\section{Rumusan Masalah}

Berdasarkan hal yang telah dipaparkan di latar belakang, rumusan masalah
yang diangkat dalam Tugas Akhir ini adalah sebagai berikut:
\begin{enumerate}
  \vspace{-0.3cm}\item{Bagaimana cara mengimplementasikan \emph{multi-tenancy} pada kluster Kubernetes?}
  \vspace{-0.3cm}\item{Apa perbedaan implementasi \emph{multi-tenancy} yang dapat digunakan pada kluster Kubernetes?}
\end{enumerate}

\section{Batasan Masalah atau Ruang Lingkup}

Batasan masalah atau ruang lingkup dari penelitian ini adalah sebagai berikut:
\begin{enumerate}
  \vspace{-0.3cm}\item{asdasdad}
\end{enumerate}

\section{Tujuan}

Tujuan dari penelitian ini adalah mengimplementasikan \emph{multi-tenancy}
pada kluster Kubernetes yang dapat digunakan.

\section{Manfaat}

Manfaat dari penelitian ini adalah 
