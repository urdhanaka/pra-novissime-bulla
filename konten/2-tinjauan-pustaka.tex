\chapter{TINJAUAN PUSTAKA}

\section{Hasil penelitian/perancangan terdahulu}

Kubernetes memiliki beberapa fitur yang memungkinkan untuk penerapan
\emph{multi-tenancy}. Salah satu fitur tersebut adalah \emph{namespace}.
\emph{Namespace} adalah fitur yang memisahkan sumber daya di dalam kluster
ke dalam beberapa bagian.

\section{Teori/Konsep Dasar}

\subsection{Kluster}

Kluster adalah konsep dasar dari Kubernetes. Kluster adalah kumpulan dari
satu atau lebih \emph{node} yang digunakan untuk menjalankan \emph{pods} yang
menjalankan aplikasi yang dikemas (\emph{containerized}). 

\subsection{\emph{Multi-tenant}}

\emph{Multi-tenant} adalah sebuah konsep dimana sebuah sistem dapat digunakan
oleh lebih dari satu pengguna atau \emph{tenant}. Dalam Kubernetes, implementasi
\emph{multi-tenant} berada di kluster sehingga sebuah kluster dapat digunakan
oleh satu atau lebih pengguna. Maka \parencite{oliva_multi-tenancy_2024}
