\chapter*{ABSTRACT}
\begin{center}
  \large
  \textbf{MULTI-TENANCY IMPLEMENTATION FOR KUBERNETES CLUSTER PROVISIONING}
\end{center}
\addcontentsline{toc}{chapter}{ABSTRACT}
% Menyembunyikan nomor halaman
\thispagestyle{empty}

\begin{flushleft}
  \setlength{\tabcolsep}{0pt}
  \bfseries
  \begin{tabular}{lc@{\hspace{6pt}}l}
  Student Name / NRP&: &Urdhanaka Aptanagi / 5025211123\\
  Department&: &Informatics Engineering FTEIC - ITS\\
  Advisor&:& 1. Royyana Muslim Ijtihadie, S.Kom., M.Kom., Ph.D.\\
  & & 2. Ary Mazharuddin Shiddiqi, S.Kom., M.Comp.Sc., Ph.D.\\
  \end{tabular}
  \vspace{4ex}
\end{flushleft}
\textbf{Abstract}

% Isi Abstrak
Kubernetes is a platform to orchestrate container and to manage the container and also able to create
a cluster from many physical device. However, Kubernetes doesn't support multi-tenancy
by default. Multi-tenancy has become a common practice for optimizing resource utilization in
cloud-native environments. This study explores the implementation of multi-tenancy for
Kubernetes cluster. By utilizing native Kubernetes tools and external tools, the study evaluates the
differences between multi-tenancy implementations in Kubernetes. A local Kubernetes cluster
using version 1.31.4 is used for demonstrations and evaluations. Multi-tenancy implementation
is done by literature study about native tools from Kubernetes website and external tools
from past research. The results show that multi-tenancy in Kubernetes cluster can enhance efficiency
and resource utilization among tenants. Other than that, multi-tenancy implementation splits the
bigger cluster into smaller clusters that is isolated for each cluster.
The study also show that multi-tenancy contributes to more cost-effective
cloud-native operations.

\vspace{2ex}
\noindent
\textbf{Keywords: \emph{cloud computing, kubernetes, multi-tenancy, resources isolation}}
