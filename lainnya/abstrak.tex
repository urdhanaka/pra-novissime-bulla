\chapter*{ABSTRAK}
\begin{center}
  \large
  \textbf{IMPLEMENTASI \emph{MULTI-TENANCY} UNTUK \emph{PROVISIONING} KLUSTER KUBERNETES}
\end{center}
\addcontentsline{toc}{chapter}{ABSTRAK}
% Menyembunyikan nomor halaman
\thispagestyle{empty}

\begin{flushleft}
  \setlength{\tabcolsep}{0pt}
  \bfseries
  \begin{tabular}{ll@{\hspace{6pt}}l}
  Nama Mahasiswa / NRP&:& Urdhanaka Aptanagi / 5025211123\\
  Departemen&:& Teknik Informatika FTEIC - ITS\\
  Dosen Pembimbing&:& 1. Nikola Tesla, S.T., M.T.\\
  & & 2. Wernher von Braun, S.T., M.T.\\
  \end{tabular}
  \vspace{4ex}
\end{flushleft}
\textbf{Abstrak}

% Isi Abstrak
Sistem terdistribusi merupakan salah satu cara untuk meningkatkan kinerja
dan keandalan sebuah sistem. Sebuah \emph{tools} yang sering digunakan dalam sistem
terdistribusi adalah Kubernetes.

\vspace{2ex}
\noindent
\textbf{Kata Kunci: \emph{multi-tenancy, virtual cluster, kubernetes}}
