\chapter*{ABSTRAK}
\begin{center}
  \large
  \textbf{IMPLEMENTASI \emph{MULTI-TENANCY} UNTUK \emph{PROVISIONING} KLASTER KUBERNETES}
\end{center}
\addcontentsline{toc}{chapter}{ABSTRAK}
% Menyembunyikan nomor halaman
\thispagestyle{empty}

\begin{flushleft}
  \setlength{\tabcolsep}{0pt}
  \bfseries
  \begin{tabular}{ll@{\hspace{6pt}}l}
  Nama Mahasiswa / NRP&:& Urdhanaka Aptanagi / 5025211123\\
  Departemen&:& Teknik Informatika FTEIC - ITS\\
  Dosen Pembimbing&:& 1. Royyana Muslim Ijtihadie, S.Kom., M.Kom., Ph.D.\\
  & & 2. Ary Mazharuddin Shiddiqi, S.Kom., M.Comp.Sc., Ph.D.\\
  \end{tabular}
  \vspace{4ex}
\end{flushleft}
\textbf{Abstrak}

% Isi Abstrak
Kubernetes adalah salah satu \emph{platform} yang digunakan untuk orkestrasi \emph{container},
mengatur jalannya \emph{container} serta membuat sebuah klaster dari banyak
perangkat fisik. Namun, Kubernetes tidak memiliki \emph{support} untuk \emph{multi-tenancy}
secara \emph{default}. \emph{Multi-tenancy} telah menjadi hal yang umum untuk mengoptimasi utilisasi
\emph{resources} dalam lingkungan \emph{cloud-native}. \emph{Multi-tenancy} pada lingkungan
\emph{cloud-native} membagi \emph{resource} yang ada agar dapat digunakan oleh banyak pengguna (\emph{multi tenant}).
Kubernetes memiliki beberapa fitur untuk \emph{multi-tenancy}, salah satunya yaitu \emph{role-based access control}
untuk membagi \emph{resource}.
Penelitian dan riset sebelumnya mengenai implementasi \emph{multi-tenancy} pada Kubernetes
Pada proposal ini, akan 

Kubernetes membagi klaster Kubernetes menjadi beberapa klaster
kecil untuk pengguna. Pembagian ini dilakukan berdasarkan \emph{resource} komputasi
yang dibutuhkan oleh pengguna. Selain itu, klaster kecil tersebut bersifat terisolasi
dari klaster kecil lainnya dan \emph{resource} lain yang bukan milik klaster tersebut.
Studi ini mengeksplor implementasi dari \emph{multi-tenancy} untuk klaster Kubernetes. Dengan menggunakan \emph{native tools}
dari Kubernetes dan \emph{external tools} lainnya, studi ini mengevaluasi perbedaan dari implementasi \emph{multi-tenancy} di Kubernetes.
Klaster Kubernetes lokal versi 1.31.4 digunakan untuk demonstrasi dan evaluasi. Implementasi
dilakukan dengan melakukan studi literatur mengenai \emph{native tools} dari Kubernetes melalui
\emph{website} resmi Kubernetes dan mengenai \emph{external tools} dari penelitian-penelitian sebelumnya.
Hasil yang didapat menunjukkan bahwa \emph{multi-tenancy} dapat meningkatkan efisiensi dan
utilisasi \emph{resources} untuk penggunanya. Selain itu, implementasi \emph{multi-tenancy} dapat membagi
klaster menjadi klaster-klaster yang kecil dengan isolasi untuk tiap klaster.
Studi ini juga menunjukkan bahwa \emph{multi-tenancy} berkontribusi untuk operasi \emph{cloud-native} yang lebih hemat biaya.

\vspace{2ex}
\noindent
\textbf{Kata Kunci: \emph{cloud computing, kubernetes, multi-tenancy, resources isolation}}
