\chapter*{LEMBAR PENGESAHAN}
\addcontentsline{toc}{chapter}{LEMBAR PENGESAHAN}
\thispagestyle{empty}

\begin{center}
  % Ubah kalimat berikut dengan judul tugas akhir
  \textbf{IMPLEMENTASI \emph{MULTI-TENANCY} UNTUK \emph{PROVISIONING} KLASTER KUBERNETES}
\end{center}

\begingroup
% Pemilihan font ukuran small
\small

\begin{center}
  % Ubah kalimat berikut dengan pernyataan untuk lembar pengesahan
  \textbf{PROPOSAL TUGAS AKHIR} \\
  Diajukan untuk memenuhi salah satu syarat memperoleh gelar
  Sarjana Komputer pada
  Program Studi S-1 Teknik Informatika \\
  Departemen Teknik Informatika \\
  Fakultas Teknologi Elektro dan Informatika Cerdas \\
  Institut Teknologi Sepuluh Nopember
\end{center}

\begin{center}
  % Ubah kalimat berikut dengan nama dan NRP mahasiswa
  Oleh: \textbf{Urdhanaka Aptanagi} \\
  NRP. 5025211123
\end{center}

\begin{center}
  Disetujui Oleh:
\end{center}

\vspace{10ex}

\begingroup
% Menghilangkan padding
\setlength{\tabcolsep}{0pt}

\noindent
\begin{tabularx}{\textwidth}{X c}
  % Ubah kalimat-kalimat berikut dengan nama dan NIP dosen pembimbing pertama
        &                 \\
  Royyana Muslim Ijtihadie, S.Kom., M.Kom., Ph.D.     &                 \\
  NIP: 19770824 200304 1 001                          & (Pembimbing)    \\
                                                      &                 \\
                                                      &                 \\
                                                      &                 \\
  % Ubah kalimat-kalimat berikut dengan nama dan NIP dosen pembimbing kedua
  Ary Mazharuddin Shiddiqi, S.Kom., M.Comp.Sc., Ph.D. &                 \\
  NIP: 19810620 200501 1 003                          & (Ko-Pembimbing) \\
\end{tabularx}
\endgroup

\vspace{\fill}

\begin{center}
  \textbf{SURABAYA} \\
  \textbf{Desember, 2024}
\end{center}
\endgroup
